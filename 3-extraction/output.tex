\documentclass[leqno]{article}
\usepackage[utf8]{inputenc}
\usepackage{graphicx}
\usepackage{dsfont}
\usepackage{amssymb}
\usepackage{amsmath}
\usepackage{amsthm}
\usepackage{float}

% (ii) Added 'urlcolor=black' to make the link black
\usepackage[colorlinks=true,linkcolor=red,citecolor=magenta,urlcolor=black]{hyperref}%
\urlstyle{same} % (ii) Makes the link font match the body text

\usepackage{mathtools}
\usepackage{enumitem}
\usepackage{dirtytalk}

\usepackage[margin=1.2in,headheight=1pt]{geometry}

\usepackage{ragged2e}
\usepackage{tensor}
\usepackage{fancyhdr}
\setlength{\parindent}{0cm}

\usepackage{empheq}
\usepackage[most]{tcolorbox}

\newtheorem*{theorem*}{Theorem}
\newtheorem{definition}{Definition}
\newtheorem*{definition*}{Definition}

\renewcommand{\marks}[1]{\marginpar{\raggedleft\textbf{[#1]}}}

% --- HEADER CONFIG ---
\pagestyle{fancy}
\fancyhead{} % Clear all default header fields
\fancyfoot{} % Clear all footer fields
\renewcommand{\headrulewidth}{0.8pt} % Set header rule width

\lhead{SAMPLE} % Left header
\chead{} % Center header
\rhead{} % Right header
% ---------------------

\begin{document}

\begin{enumerate}[label=\textbf{\arabic*.}, leftmargin=*, itemsep=0.75ex]

\item
% Answer: 1 (Letter: B)

PQRS is a quadrilateral, labelled anticlockwise.

Which one of the following is a sufficient but not necessary condition for PQRS to be a parallelogram?

\vspace{0.75cm}

\begin{tabular}{@{}ll@{}}
\textbf{A} & $PR$ and $QS$ bisect each other \\[8pt]
\textbf{B} & $PQ=QR=RS=SP$ \\[8pt]
\textbf{C} & $\angle P + \angle R = 180^{\circ}$ and $\angle Q + \angle S = 180^{\circ}$ \\[8pt]
\textbf{D} & $PR$ bisects $QS$ \\[30pt]
\end{tabular}

\hrulefill

\item
% Answer: 2 (Letter: C)

A student chooses two distinct real numbers $x$ and $y$ with $0<x<y<1$.

The student then attempts to draw a triangle $ABC$ with:

\[
\begin{array}{c}
AB = 1, \qquad \cos A = x, \qquad \cos B = y
\end{array}
\]

Which of the following statements is/are correct?

\[
\begin{array}{@{}l l@{}}
\textbf{I:} & \text{For some choice of $x$ and $y$, there is no such triangle.} \\[2pt]
\textbf{II:} & \text{For some choice of $x$ and $y$, there is exactly one such triangle.} \\[2pt]
\textbf{III:} & \text{For some choice of $x$ and $y$, there are exactly two such triangles.}
\end{array}
\]

(Note that congruent triangles are considered to be the same.)

\vspace{0.75cm}

\begin{tabular}{@{}ll@{}}
\textbf{A} & None of them \\[2pt]
\textbf{B} & I only \\[2pt]
\textbf{C} & II only \\[2pt]
\textbf{D} & III only \\[2pt]
\textbf{E} & I and II only \\[2pt]
\textbf{F} & I and III only \\[2pt]
\textbf{G} & II and III only \\[2pt]
\textbf{H} & I, II and III \\[12pt]
\end{tabular}

\hrulefill

\item
% Answer: 0 (Letter: A)

Let $n>1$ be an integer. Consider the property $P(n)$:

\[
\begin{array}{c}
\text{For all integers $a,b$, if $n\mid ab$ then $n\mid a$ or $n\mid b$.}
\end{array}
\]

Which of the following is correct?

\vspace{0.75cm}

\begin{tabular}{@{}ll@{}}
\textbf{A} & $P(n)$ holds if and only if $n$ is prime \\[8pt]
\textbf{B} & $P(n)$ holds if and only if $n$ is squarefree \\[8pt]
\textbf{C} & $P(n)$ is sufficient but not necessary for $n$ to be prime \\[8pt]
\textbf{D} & $P(n)$ holds if and only if $n$ is a prime power \\[8pt]
\textbf{E} & $P(n)$ holds if and only if $n$ has at most two distinct prime factors \\[30pt]
\end{tabular}

\hrulefill

\item
% Answer: 0 (Letter: A)

Let $a, b, n$ be positive integers and let $n = p_1^{e_1} p_2^{e_2} \cdots p_k^{e_k}$ be the prime factorization of $n$. Consider the statement
\begin{equation}\tag{$\ast$}
\text{$n$ divides $ab$.}
\end{equation}
The condition $C$ is:
\[
\begin{array}{c}
\text{For every prime $p$ dividing $n$, either $p\mid a$ or $p\mid b$.}
\end{array}
\]

This condition $C$ is:

\vspace{0.75cm}

\begin{tabular}{@{}ll@{}}
\textbf{A} & necessary but not sufficient for ($\ast$) \\[6pt]
\textbf{B} & sufficient but not necessary for ($\ast$) \\[6pt]
\textbf{C} & necessary and sufficient for ($\ast$) \\[6pt]
\textbf{D} & neither necessary nor sufficient for ($\ast$) \\[24pt]
\end{tabular}

\hrulefill

\item
% Answer: 0 (Letter: A)

Let $P(n)$ be a statement about integers with the following properties:

\[
\begin{array}{l}
\text{$P(1)$ and $P(2)$ are true;} \\[2pt]
\text{for every integer $n\ge 1$, the conjunction $P(n)\wedge P(n+1)$ implies $P(n+2)$.}
\end{array}
\]

What is the strongest conclusion that necessarily follows?

\vspace{0.75cm}

\begin{tabular}{@{}ll@{}}
\textbf{A} & $P(n)$ holds for all $n \ge 1$ \\[6pt]
\textbf{B} & $P(n)$ holds for all sufficiently large $n$, but not necessarily for $n=1$ \\[6pt]
\textbf{C} & $P(n)$ holds for infinitely many $n$, but not necessarily for all $n$ \\[6pt]
\textbf{D} & No general conclusion can be drawn beyond $P(1)$ and $P(2)$ \\[24pt]
\end{tabular}

\hrulefill

\end{enumerate}
\end{document}
