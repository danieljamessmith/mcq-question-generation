\documentclass[leqno]{article}
\usepackage[utf8]{inputenc}
\usepackage{graphicx}
\usepackage{dsfont}
\usepackage{amssymb}
\usepackage{amsmath}
\usepackage{amsthm}
\usepackage{float}

% (ii) Added 'urlcolor=black' to make the link black
\usepackage[colorlinks=true,linkcolor=red,citecolor=magenta,urlcolor=black]{hyperref}%
\urlstyle{same} % (ii) Makes the link font match the body text

\usepackage{mathtools}
\usepackage{enumitem}
\usepackage{dirtytalk}

\usepackage[margin=1.2in,headheight=1pt]{geometry}

\usepackage{ragged2e}
\usepackage{tensor}
\usepackage{fancyhdr}
\setlength{\parindent}{0cm}

\usepackage{empheq}
\usepackage[most]{tcolorbox}

\newtheorem*{theorem*}{Theorem}
\newtheorem{definition}{Definition}
\newtheorem*{definition*}{Definition}

% --- HEADER CONFIG ---
\pagestyle{fancy}
\fancyhead{} % Clear all default header fields
\fancyfoot{} % Clear all footer fields
\renewcommand{\headrulewidth}{0.8pt} % Set header rule width

\lhead{SAMPLE} % Left header
\chead{} % Center header
\rhead{} % Right header
% ---------------------

\begin{document}

\vspace*{1cm}
\begin{enumerate}[label=\textbf{\arabic*.}, leftmargin=*, itemsep=0.75ex]

\item We say a real function $f(x)$ is \textit{continuous} at a point $x_0 \in \mathbb{R}$ if


\[
\begin{array}{c}
\text{for all $\varepsilon>0$ there exists $\delta>0$ such that for all $x\in\mathbb{R}$,} \\[2pt]
\text{if $|x-x_0|<\delta$ then $|f(x)-f(x_0)|<\varepsilon$}
\end{array}
\]


Which one of the following statements is true if and only if
$f$ is \textbf{not} continuous at $x_0$?

\vspace{0.75cm}

\begin{tabular}{@{}ll@{}}
\textbf{A} & There exists $\varepsilon>0$ such that for every $\delta>0$ there exists $x$ with 
$|x-x_0|<\delta$ and $|f(x)-f(x_0)|\ge \varepsilon$ \\[8pt]
\textbf{B} & For every $\varepsilon>0$ and every $\delta>0$ there exists $x$ with 
$|x-x_0|<\delta$ and $|f(x)-f(x_0)|\ge \varepsilon$ \\[8pt]
\textbf{C} & There exists $\varepsilon>0$ and $\delta>0$ such that for all $x$ with 
$|x-x_0|<\delta$ we have $|f(x)-f(x_0)|\ge \varepsilon$ \\[8pt]
\textbf{D} & There exists $\varepsilon>0$ such that for every $\delta>0$ we have $|x-x_0|\ge \delta$ or $|f(x)-f(x_0)|\ge \varepsilon$ \\[8pt]
\textbf{E} & For every $\varepsilon>0$ and every $\delta>0$ there exists $x$ with 
$|x-x_0|<\delta$ and $|f(x)-f(x_0)|< \varepsilon$ \\[8pt]
\textbf{F} & There exists $\delta>0$ such that for all $\varepsilon>0$ and all $x$ with 
$|x-x_0|<\delta$ we have $|f(x)-f(x_0)|<\varepsilon$ \\[8pt]
\textbf{G} & For every $\varepsilon>0$ there exists $\delta>0$ such that there exists $x$ with 
$|x-x_0|<\delta$ and $|f(x)-f(x_0)|\ge \varepsilon$ \\[8pt]
\textbf{H} & There exists $\varepsilon>0$ such that for every $\delta>0$ if $|x-x_0|<\delta$ then $|f(x)-f(x_0)|\ge \varepsilon$ \\[30pt]

\end{tabular}

\hrulefill
\vspace{30pt}

\item We say a sequence of real numbers $x_n$ is \textit{a Cauchy sequence} if


\[
\begin{array}{c}
\text{for all $\varepsilon>0$ there exists a positive integer $N \in\mathbb{N}$ such that}  \\[2pt]
\text{for all $n,m \geq N$ we have $|x_n - x_m | < \varepsilon$}
\end{array}
\]


Which one of the following statements is true if and only if
$x_n$ is \textbf{not} a Cauchy sequence?

\vspace{0.75cm}

\begin{tabular}{@{}ll@{}}
\textbf{A} & There exists $\varepsilon>0$ such that for every $N\in\mathbb{N}$ there exist 
$m,n\ge N$ with $|x_m-x_n|\ge \varepsilon$ \\[8pt]
\textbf{B} & For every $\varepsilon>0$ there exists $N\in\mathbb{N}$ and there exists 
$m,n\ge N$ with $|x_m-x_n|\ge \varepsilon$ \\[8pt]
\textbf{C} & There exists $N\in\mathbb{N}$ such that for all $\varepsilon>0$ and all $m,n\ge N$
we have $|x_m-x_n|\ge \varepsilon$ \\[8pt]
\textbf{D} & For every $\varepsilon>0$ and every $N\in\mathbb{N}$ there exists $m,n\ge N$
with $|x_m-x_n|<\varepsilon$ \\[8pt]
\textbf{E} & There exists $\varepsilon>0$ and there exists $N\in\mathbb{N}$ such that 
for all $m,n\ge N$ we have $|x_m-x_n|\ge \varepsilon$ \\[8pt]
\textbf{F} & For every $N\in\mathbb{N}$ there exists $\varepsilon>0$ such that for all $m,n\ge N$
we have $|x_m-x_n|\ge \varepsilon$ \\[30pt]
\end{tabular}


\hrulefill
\newpage

\vspace*{1cm}

\item Let $f(x),\, g(x)$ and $h(x)$ be real functions and consider the statement

\begin{equation}\tag{$\ast$}
\text{If $x>0$ and $f(x)\le 0$ then $g(x)\neq 1$ or $h(x)\ge -1$.}\\[5pt]
\end{equation}


Which of the following statements is the \textit{contrapositive} of ($\ast$)?

\vspace{0.75cm}

\begin{tabular}{@{}ll@{}}
\textbf{A} & If $g(x) = 1$ and $h(x) < -1$ then $x \le 0$ or $f(x) > 0$. \\[8pt]
\textbf{B} & If $g(x) = 1$ or $h(x) < -1$ then $x \le 0$ or $f(x) > 0$. \\[8pt]
\textbf{C} & If $x \le 0$ or $f(x) > 0$ then $g(x) = 1$ and $h(x) < -1$. \\[8pt]
\textbf{D} & If $g(x) \neq 1$ or $h(x) \ge -1$ then $x > 0$ and $f(x) \le 0$. \\[8pt]
\textbf{E} & If $x > 0$ and $f(x) \le 0$ then $g(x) = 1$ and $h(x) < -1$. \\[30pt]
\end{tabular}

\hrulefill
\vspace*{30pt}

\item Let $f(x),\, g(x),\, h(x)$ and $k(x)$ be real functions and consider the statement


\begin{equation}\tag{$\ast$}
\text{If $0<x<1$ and $f(x) > g(x)$, then $h(x) = 0$ or $k(x) \ge 5$.}\\[5pt]
\end{equation}


Which of the following statements is the \textit{contrapositive} of $(\ast)$?

\vspace{0.75cm}

\begin{tabular}{@{}ll@{}}

\textbf{A} & If $h(x) \neq 0$ or $k(x) < 5$ then $x \le 0$ or $x \ge 1$ or $f(x) \le g(x)$. \\[8pt]
\textbf{B} & If $h(x) \neq 0$ and $k(x) < 5$ then $x \le 0$ or $x \ge 1$ and $f(x) \le g(x)$. \\[8pt]
\textbf{C} & If $h(x) \neq 0$ and $k(x) < 5$ then $x \le 0$ or $x \ge 1$ or $f(x) \le g(x)$. \\[8pt]
\textbf{D} & If $x \le 0$ or $x \ge 1$ or $f(x) \le g(x)$ then $h(x) \neq 0$ and $k(x) < 5$. \\[8pt]
\textbf{E} & If $h(x) = 0$ or $k(x) \ge 5$ then $0<x<1$ and $f(x) > g(x)$. \\[30pt]
\end{tabular}

\hrulefill
\newpage
\vspace*{1cm}
\item
Consider the following statement about irrational numbers:
\begin{equation}\tag{$\ast$}
\text{If $x$ and $y$ are irrational numbers, then $x + y$ is irrational.}
\end{equation}

Which of the following pair(s) $(x,y)$ provide(s) a \textit{counterexample} to $(\ast)$?

\[
\begin{array}{@{}l l@{}}
\textbf{I:} & x = \sqrt{2},\ y = \sqrt{2} \\[2pt]
\textbf{II:} & x = \sqrt{2},\ y = 3 \\[2pt]
\textbf{III:} & x = 1 + \sqrt{2},\ y = 1 - \sqrt{2}
\end{array}
\]

\vspace{0.75cm}

\begin{tabular}{@{}ll@{}}
\textbf{A} & none of them \\[2pt]
\textbf{B} & I only \\[2pt]
\textbf{C} & II only \\[2pt]
\textbf{D} & III only \\[2pt]
\textbf{E} & I and II only \\[2pt]
\textbf{F} & I and III only \\[2pt]
\textbf{G} & II and III only \\[2pt]
\textbf{H} & I, II and III \\[30pt]
\end{tabular}

\hrulefill

\vspace*{30pt}

\item
Consider the following statement about a polynomial $p(x)$ with real coefficients:
\begin{equation}\tag{$\ast$}
\text{If $p(n)$ is even for every integer $n$, then every coefficient of $p(x)$ is even.}
\end{equation}

Which one of the following is a \textit{counterexample} to $(\ast)$?

\vspace{0.5cm}

\begin{tabular}{@{}ll@{}}
\textbf{A} & $p(x) = x^{2} + x$ \\[6pt]
\textbf{B} & $p(x) = 2x^{2} + 4$ \\[6pt]
\textbf{C} & $p(x) = 2x^{2} + x$ \\[6pt]
\textbf{D} & $p(x) = 4x^{3} + 2x$ \\[30pt]
\end{tabular}

\hrulefill


\end{enumerate}
\end{document}