\begin{document}
\vspace*{10pt}
\begin{enumerate}[label=\textbf{\arabic*.}, leftmargin=*, itemsep=0.75ex]

\item
% Answer: 1 (Letter: B)

PQRS is a quadrilateral, labelled anticlockwise.

Which one of the following is a \textbf{sufficient} but \textbf{not necessary} condition for PQRS to be a parallelogram?

\vspace{0.75cm}

\begin{tabular}{@{}ll@{}}
\textbf{A} & $PR$ and $QS$ bisect each other \\[8pt]
\textbf{B} & $PQ=QR=RS=SP$ \\[8pt]
\textbf{C} & $\angle P + \angle R = 180^{\circ}$ and $\angle Q + \angle S = 180^{\circ}$ \\[8pt]
\textbf{D} & $PR$ bisects $QS$ \\[30pt]
\end{tabular}

\hrulefill
\vspace{10pt}
\item


% Answer: 2 (Letter: C)
A student chooses two distinct real numbers $x$ and $y$ with $0<x<y<1$

The student then attempts to draw a triangle $ABC$ with:

\[
\begin{array}{c}
AB = 1 \\  \cos A = x \\ \cos B = y
\end{array}
\]

Which of the following statements is/are correct?

\[
\begin{array}{@{}l l@{}}
\textbf{I:} & \text{For some choice of $x$ and $y$, there is no such triangle.} \\[2pt]
\textbf{II:} & \text{For some choice of $x$ and $y$, there is exactly one such triangle.} \\[2pt]
\textbf{III:} & \text{For some choice of $x$ and $y$, there are exactly two such triangles.}
\end{array}
\]

(Note that congruent triangles are considered to be the same.)

\vspace{0.75cm}

\begin{tabular}{@{}ll@{}}
\textbf{A} & None of them \\[2pt]
\textbf{B} & I only \\[2pt]
\textbf{C} & II only \\[2pt]
\textbf{D} & III only \\[2pt]
\textbf{E} & I and II only \\[2pt]
\textbf{F} & I and III only \\[2pt]
\textbf{G} & II and III only \\[2pt]
\textbf{H} & I, II and III \\[12pt]
\end{tabular}

\hrulefill
\newpage
\vspace*{10pt}

\item
% Answer: 0 (Letter: A)


We say an integer $x$ \textit{divides} an integer $y$ and write $x \,|\, y $ if there exists an integer $k$ such that $y = kx$

Let $a, b, n$ be positive integers. Let $(\ast)$ the statement
\begin{equation}\tag{$\ast$}
n \,|\, ab
\end{equation}
Let C be the statement:
\[
\begin{array}{c}
\text{For every prime $p$ with $p \,|\, n$, either $p\mid a$ or $p\mid b$}
\end{array}
\]

This condition $C$ is:

\vspace{0.75cm}

\begin{tabular}{@{}ll@{}}
\textbf{A} & \textbf{necessary} but \textbf{not sufficient} for ($\ast$) \\[6pt]
\textbf{B} & \textbf{sufficient} but \textbf{not necessary} for ($\ast$) \\[6pt]
\textbf{C} & \textbf{necessary} and \textbf{sufficient} for ($\ast$) \\[6pt]
\textbf{D} & \textbf{neither necessary nor sufficient} for ($\ast$) \\[24pt]
\end{tabular}

\hrulefill
\vspace{10pt}
\item
% Answer: 0 (Letter: A)

Let $P(n)$ be a statement about integers with the following properties:

\[
\begin{array}{l}
\text{$P(1)$ and $P(2)$ are true;} \\[8pt]
\text{for all } n\geq1 \text{ the following statement is true:}\\[2pt]
\qquad \left(P(n) \textbf{ and } P(n+1)\right) \text{ is \textbf{sufficient }for } P(n+2)
\end{array}
\]

What is the strongest conclusion that necessarily follows?

\vspace{0.75cm}

\begin{tabular}{@{}ll@{}}
\textbf{A} & $P(n)$ is true for all $n \ge 1$ \\[6pt]
\textbf{B} & $P(n)$ is true for all sufficiently large $n$, but not necessarily for all $n$  \\[6pt]
\textbf{C} & $P(n)$ is true for infinitely many $n$, but not necessarily for all $n$ \\[6pt]
\textbf{D} & $P(n)$ is true for all odd $n$ but not necessarily for all even $n$ \\[6pt]
\textbf{E} & $P(n)$ is true for all even $n$ but not necessarily for all odd $n$ \\[6pt]
\textbf{F} & No general conclusion can be drawn beyond $P(1)$ and $P(2)$ \\[24pt]
\end{tabular}

\hrulefill

\end{enumerate}
\end{document}