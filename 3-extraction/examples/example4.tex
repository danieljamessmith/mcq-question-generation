\documentclass[leqno]{article}
\usepackage[utf8]{inputenc}
\usepackage{graphicx}
\usepackage{dsfont}
\usepackage{amssymb}
\usepackage{amsmath}
\usepackage{amsthm}
\usepackage{float}

% (ii) Added 'urlcolor=black' to make the link black
\usepackage[colorlinks=true,linkcolor=red,citecolor=magenta,urlcolor=black]{hyperref}%
\urlstyle{same} % (ii) Makes the link font match the body text

\usepackage{mathtools}
\usepackage{enumitem}
\usepackage{dirtytalk}

\usepackage[margin=1.2in,headheight=1pt]{geometry}

\usepackage{ragged2e}
\usepackage{tensor}
\usepackage{fancyhdr}
\setlength{\parindent}{0cm}

\usepackage{empheq}
\usepackage[most]{tcolorbox}

\newtheorem*{theorem*}{Theorem}
\newtheorem{definition}{Definition}
\newtheorem*{definition*}{Definition}

\renewcommand{\marks}[1]{\marginpar{\raggedright\textbf{[#1]}}}

% --- HEADER CONFIG ---
\pagestyle{fancy}
\fancyhead{} % Clear all default header fields
\fancyfoot{} % Clear all footer fields
\renewcommand{\headrulewidth}{0.8pt} % Set header rule width

\lhead{India B} % Left header
\chead{Further Mechanics} % Center header
\rhead{\href{https://danieljsmith.org}{danieljsmith.org}} % Right header
% ---------------------

\begin{document}

\vspace*{0.5cm}
\subsection*{Question 1}
\vspace{10pt}

A car of mass $1200\text{ kg}$ moves in a straight line along a horizontal road at a constant speed $U\text{ms}^{-1}$. The resistance to the motion of the car is a constant force of magnitude $500\text{ N}$.

The engine of the car is working at a constant rate of $20 \text{ kW}$.


\begin{enumerate}[label=\textbf{(\alph*)}]

\item Find the value of $U$ \marks{3 Marks}
\end{enumerate}
The car now pulls a trailer of mass $800\text{ kg}$ in a straight line along the road using a tow rope which is parallel to the direction of motion. The resistance to the motion of the car is again a constant force of magnitude $500\text{ N}$. The resistance to the motion of the trailer is a constant force of magnitude $400\text{ N}$.
The engine of the car is working at a constant rate of $20\text{ kW}$.
The tow rope is modelled as being light and inextensible. \\

Using the model,
\begin{enumerate}[label=\textbf{(\alph*)}, resume]
\item find the tension in the tow rope at the instant when the speed of the car is $\dfrac{25}{3}\text{ ms}^{-1}$ \marks{5 Marks}
\end{enumerate}

\vspace{20pt}
\hrulefill
\vspace{20pt}

\subsection*{Question 2}
\vspace{10pt}

A particle $P$ of mass $2m$ is moving in a straight line with speed $4u$ on a smooth horizontal plane. It collides directly with a particle $Q$ of mass $m$ that is moving on the plane with speed $3u$ in the opposite direction to $P$. \\[3pt]

The coefficient of restitution between $P$ and $Q$ is $e$, where $e > \dfrac{5}{7}$.  \\[3pt]

\begin{enumerate}[label=\textbf{(\alph*)}]
\item Show that the speed of $Q$ immediately after the collision is $\dfrac{14e + 5}{3}u$ \marks{6 Marks} \\[3pt]
\end{enumerate}

After the collision $Q$ hits a smooth fixed vertical wall that is perpendicular to the direction of motion of $Q$. The coefficient of restitution between $Q$ and the wall is $f$. \\[3pt]

\begin{enumerate}[label=\textbf{(\alph*)}, resume]
\item Find, in terms of $e$, the set of values of $f$ for which there will be a second collision between $P$ and $Q$. \marks{4 Marks}
\end{enumerate}

\vspace{20pt}
\hrulefill




\end{document}