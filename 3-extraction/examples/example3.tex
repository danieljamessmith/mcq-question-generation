\begin{document}

\begin{enumerate}[label=\textbf{\arabic*.}, leftmargin=*, itemsep=0.75ex]
\item
Let $p$ be a real constant. Consider the equation \[\sin x \cos x = p\] on $0 \le x < 2\pi$. Which of the following statements is/are true?

\[
\begin{array}{@{}l l@{}}
\textbf{I:} & \text{If } |p|<\tfrac12 \text{ then the equation has exactly 4 distinct solutions.} \\[2pt]
\textbf{II:} & \text{If the equation has exactly 2 distinct solutions, then } |p|=\tfrac12 .
\end{array}
\]

\vspace{0.75cm}

\begin{tabular}{@{}ll@{}}
\textbf{A} & None of them \\[2pt]
\textbf{B} & I only \\[2pt]
\textbf{C} & II only \\[2pt]
\textbf{D} & I and II
\end{tabular} \\[2pt]

\hrulefill
\item

  For a positive integer \(n\), consider the identity
  \[
    \sum_{k=0}^{n-1}\cos\!\left(\theta+\frac{k\pi}{2}\right)=0
  \]
  for all real $\theta$. Which of the following is a \textbf{necessary and sufficient} condition on \(n\)
  for the identity to hold?

\vspace{0.75cm}

\begin{tabular}{@{}ll@{}}
\textbf{A} & $n$ is odd \\[2pt]
\textbf{B} & $n$ is even \\[2pt]
\textbf{C} & $n$ is a multiple of 3 \\[2pt]
\textbf{D} & $n$ is a multiple of 4 \\[2pt]
\textbf{E} & $n$ is 1 more than a multiple of 4 \\[2pt]
\textbf{F} & $n$ is 2 more than a multiple of 4 \\[2pt]
\end{tabular} \\[2pt]

\hrulefill

\item   We say a sequence of real numbers $x_n$ converges to a real number $L$
  if for all $\varepsilon>0$ there exists a positive integer $N\in\mathbb{N}$ such that if
  $n\ge N$ then $|x_n-L|<\varepsilon$.  \\[5pt]
  Which one of the following statements is true if and only if
  $x_n$ does \textbf{not} converge to $L$?

\vspace{0.75cm}

\begin{tabular}{@{}ll@{}}
\textbf{A} & There exists $\varepsilon>0$ such that for every $N\in\mathbb{N}$ there exists $n\ge N$ with $|x_n-L|\ge \varepsilon$ \\[5pt]
\textbf{B} & For every $\varepsilon>0$ there exists $N\in\mathbb{N}$ and there exists $n\ge N$ with $|x_n-L|\ge \varepsilon$ \\[5pt]
\textbf{C} & There exists $N\in\mathbb{N}$ such that for every $\varepsilon>0$ and every $n\ge N$ we have $|x_n-L|\ge \varepsilon$ \\[5pt]
\textbf{D} & For every $\varepsilon>0$ and for every $N\in\mathbb{N}$ there exists $n\ge N$ with $|x_n-L|<\varepsilon$ \\[5pt]
\textbf{E} & There exists $\varepsilon>0$ and there exists $N\in\mathbb{N}$ such that for all $n\ge N$ we have $|x_n-L|\ge \varepsilon$ \\[5pt]
\textbf{F} & For every $N\in\mathbb{N}$ there exists $\varepsilon>0$ such that for all $n\ge N$ we have $|x_n-L|\ge \varepsilon$ \\[5pt]
\textbf{G} & There exists $\varepsilon>0$ such that there exists $N\in\mathbb{N}$ and for all $n\ge N$ we have $|x_n-L|<\varepsilon$ \\[5pt]
\textbf{H} & For every $\varepsilon>0$ there exists $N\in\mathbb{N}$ such that there exists $n\ge N$ with $|x_n-L|<\varepsilon$ \\[12pt]
\end{tabular}

\hrulefill

\end{enumerate}
\end{document}