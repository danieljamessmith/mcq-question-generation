\begin{document}

\vspace*{0.5cm}
\subsubsection*{Question 1}

Let $x_n \in \mathbb{R}$ be a convergent sequences of real numbers. That is, suppose there exists some $L \in\mathbb{R}$ such that $x_n\xrightarrow[]{}L$. \\

Prove that $x_n$ is a Cauchy sequence. \marks{5 Marks}\\[5pt]

\textit{Hint:} Use the triangle inequality to show that $|x_n-x_m| \leq |x_n - L| + |x_m - L|$. Why is this useful? \\[20pt]

\hrulefill
\vspace{10pt}

\subsubsection*{Question 2}

Prove that the set of rational numbers $\mathbb{Q}$ is not \textit{complete} by giving an example of a Cauchy sequence $q_n\in\mathbb{Q}$ that does not converge to an element of $\mathbb{Q}$. \marks{10 Marks} \\[5pt]

    \textit{Hint:} Build a sequence of rational approximations to $\sqrt{2}\approx 1.414213\dots$ \\ E.g. $q_1 = 1,\,q_2=1.4,\,q_3=1.41,\,q_4=1.414,\, q_5=\dots$ \\ Find an explicit formula for $q_n$. Why is this sequence Cauchy? Why does it not converge in $\mathbb{Q}$? \\[20pt]

\hrulefill
\vspace{10pt}
\subsubsection*{Question 3}

For the following sequences $x_n \in \mathbb{R}$, find 
$\limsup\limits_{n\to\infty}x_n $, $\liminf\limits_{n\to\infty}x_n$ 
and state if $x_n$ is a Cauchy sequence: \\[5pt]

\begin{enumerate}[label=\textbf{(\alph*)}]
    % Use \dfrac ("display-style fraction") to make them larger
    \item \quad$x_n = \dfrac{-4 + (-1)^n}{2 + \dfrac{1}{3^n}}$ \\[10pt]

    \item \quad$x_n = \dfrac{(-1)^n}{n} + \dfrac{2}{3^n}$ \\[8pt]

    \item \quad$x_n = \dfrac{1}{(-1)^n+2+n^{-4}}$
\end{enumerate} \marks{15 Marks}

\newpage
\subsubsection*{Question 4}

For an arbitrary point $\mathbf{x} \in \mathbb{R}^k$ define the norms $||\mathbf{x}||_1$ and $||\mathbf{x}||_2$ as given in lectures. \\ [5pt]

    True or False? For all $\mathbf{x} \in \mathbb{R}^k$ we have \\
\[
||\mathbf{x}||_2\leq||\mathbf{x}||_1 
\] \vspace{2pt}

Either give a proof or provide a counterexample. \marks{10 Marks} \\[7.5pt]

\hrulefill
\vspace{7.5}
\subsubsection*{Question 5}

\begin{enumerate}[label=\textbf{(\alph*)}]

  \item
  Complete the following statement of the Bolzano--Weierstrass theorem for sequences in $\mathbb{R}$ and its generalisation to sequences in $\mathbb{R}^k$:

  \begin{theorem*}[Bolzano--Weierstrass] \
  
    If $(x_n)$ is a bounded sequence in $\mathbb{R}$, then $\dots$

    More generally, if $(\mathbf{x}_n)$ is a bounded sequence in $\mathbb{R}^k$, then $\ddots$
  \end{theorem*}
  \marks{5 Marks}

  \item
  Give an example of an unbounded sequence $(x_n)$ in $\mathbb{R}$ that has a convergent subsequence $(x_{n_k})$.
  Thus the converse of the Bolzano--Weierstrass theorem does not hold.
  \marks{10 Marks}

  \item
  Decide which of the following sequences have a convergent subsequence $(x_{n_k})$.
  Note that unboundedness is insufficient to deduce the non-existence of a convergent subsequence by part~\textbf{(b)}.
  Justify your answers.

  \begin{enumerate}[label=\textbf{(\roman*)}]
    \item $x_n = 4\cos n$
    \item $x_n = \dfrac{n\sin n}{1+n^2}$
    \item $x_n = n + e^{-n}$
    \item $x_n = \dfrac{\log n}{n}$ 
    \item $x_n = \left(\log\log n, \, \dfrac{1}{n^2}\right)$
    \item $x_n = \left( \dfrac{1000}{1+n},\, e^{-n},\, \dfrac{n^2}{n!},\, 3.2\right)$
    \item $x_n \in \mathbb{R}^3$ is defined by
      \[
        x_n =
        \begin{cases}
          (n,0,0), & n \equiv 1 \pmod 3,\\
          (0,n,0), & n \equiv 2 \pmod 3,\\
          (0,0,n), & n \equiv 0 \pmod 3.
        \end{cases}
      \]
      For example,
      $x_1=(1,0,0)$, $x_2=(0,2,0)$, $x_3=(0,0,3)$, $x_4=(4,0,0)$, $x_5=(0,5,0)$, and so on.
  \marks{25 Marks}
  \end{enumerate}

\end{enumerate}



\end{document}